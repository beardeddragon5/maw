% !TEX root = ../Masterdatei.tex
%HEADER

%<---!!!!!!!!!!!!!!! MAKRO-DEFIONITIONEN; BITTE NICHT VERAENDERN !!!!!!!!!!
%<--- ARBEIT EINSEITIG
\def\makroEinseitig{
  %KOMA-Script-Klasse: scrreprt
  %deutsches Design, Schriftgröße 12, DIN A4
  %Literaturverzeichnis und Index in Inhaltsverzerzeichnis einbinden
  \documentclass[12pt,a4paper,listof=totoc,oneside]{scrreprt}
  %Seitenspiegel einstellen
  \usepackage[a4paper]{geometry}
  \geometry{a4paper,left=30mm,right=25mm,
  bottom=20mm,top=15mm,bindingoffset=2mm,
  includehead,includefoot}
}
% ARBEIT EINSEITIG --->

\def\makroZweiseitig{
  %<--- ARBEIT ZWEISEITIG
  %KOMA-Script-Klasse: scrreprt
  %deutsches Design, zweiseitig
  %Literaturverzeichnis und Index in Inhaltsverzerzeichnis einbinden
  \documentclass[12pt,a4paper,listof=totoc,twoside, headsepline]{scrreprt}
  \usepackage[a4paper]{geometry}
  \geometry{a4paper,left=25mm,right=25mm,
  bottom=20mm,top=15mm,bindingoffset=2mm,
  includehead,includefoot}
}
% ARBEIT ZWEISEITIG --->
\def\Article{
  \documentclass[12pt,a4paper,listof=totoc,twoside, headsepline, notitlepage]{scrartcl}
  \usepackage[a4paper]{geometry}
  \geometry{a4paper,left=25mm,right=25mm,
  bottom=20mm,top=15mm,bindingoffset=2mm,
  includehead,includefoot}
}

%<--- Einstellungen Kopfzeile
\def\makroFH-Kopfzeilenstil{
  \pagestyle{scrheadings}
  \setheadsepline{0.4pt}
  \pagestyle{scrheadings}
  \renewcommand*{\chapterpagestyle}{scrheadings}
}
%Einstellungen Kopfzeile --->
%!!!!!!!!!!!!!!! MAKRO-DEFIONITIONEN; BITTE NICHT VERAENDERN !!!!!!!!!!--->


%AUSWAHL: TEXT EINSEITIG (ja/nein)
\Article
%\makroZweiseitig

%schalte Umlaute frei
\usepackage[ngerman]{babel}
%passende Codierung
\usepackage[utf8]{inputenc}
%Seitenspiegel einzustellen
\usepackage[a4paper]{geometry}
%Mathepaket
\usepackage{amsmath}
%Symbole
\usepackage{amssymb}
%griechische Symbole
\usepackage{upgreek}
%weitere Symbole
\usepackage{pxfonts}
% Phonetischen Alphabete für LaTeX
\usepackage{tipa}
%farbige Schriften
\usepackage{color}
\usepackage{scrhack}
%Bilder fixieren
\usepackage{float}
%Code formatierung
\usepackage[dvipsnames]{xcolor}
\usepackage{listings}
%Grafiken einbinden
\usepackage{graphicx}
% Kopf- und Fußzeilen
\usepackage[automark,standardstyle,markusedcase]{scrpage2}
% deutsche Überschriften
\usepackage[ngerman]{translator}
% Kopfzeilenabstand festlegen
\setlength{\headheight}{10mm}
%Abb. statt Abbildung
\usepackage{caption3}

\usepackage[style=authoryear]{biblatex}
% \bibliographystyle{dcu}
\usepackage[babel,german=quotes]{csquotes}

\usepackage{rotating}
\usepackage[colorlinks=false]{hyperref}

\addto\captionsngerman{
\renewcommand{\figurename}{Abb.}
\renewcommand{\tablename}{Tab.}
}
%Glossar-Pakage
\usepackage[
nonumberlist, %keine Seitenzahlen anzeigen
acronym,      %ein Abkürzungsverzeichnis erstellen
toc]          %Einträge im Inhaltsverzeichnis
{glossaries}
% \usepackage{cite}
%Glossar einschalten
\makeglossaries

\addbibresource{Bibliothek.bib}

\definecolor{lightlightgray}{gray}{0.95}

\lstdefinestyle{Common}
{
    basicstyle=\scriptsize\ttfamily\null,
    numbers=left,
    numbersep=1em,
    frame=single,
    framesep=\fboxsep,
    framerule=\fboxrule,
    xleftmargin=\dimexpr\fboxsep+\fboxrule,
    xrightmargin=\dimexpr\fboxsep+\fboxrule,
    breaklines=true,
    breakindent=0pt,
    tabsize=5,
    columns=flexible,
    showstringspaces=false,
    backgroundcolor = \color{lightlightgray},
    captionpos=b,% or t for top (default)
    abovecaptionskip=0.5\smallskipamount,   % there is also belowcaptionskip
}

% !TEX root = ../Masterdatei.tex

\newcommand\YAMLcolonstyle{\color{red}\mdseries}
\newcommand\YAMLkeystyle{\color{black}\bfseries}
\newcommand\YAMLvaluestyle{\color{blue}\mdseries}

\makeatletter

% here is a macro expanding to the name of the language
% (handy if you decide to change it further down the road)
\newcommand\language@yaml{yaml}

\expandafter\expandafter\expandafter\lstdefinelanguage
\expandafter{\language@yaml}
{
  keywords={true,false,null,y,n},
  keywordstyle=\color{darkgray}\bfseries,
  % basicstyle=\YAMLkeystyle,                                 % assuming a key comes first
  sensitive=false,
  comment=[l]{\#},
  morecomment=[s]{/*}{*/},
  commentstyle=\color{purple}\ttfamily,
  stringstyle=\YAMLvaluestyle\ttfamily,
  moredelim=[l][\color{orange}]{\&},
  moredelim=[l][\color{magenta}]{*},
  moredelim=**[il][\YAMLcolonstyle{:}\YAMLvaluestyle]{:},   % switch to value style at :
  morestring=[b]',
  morestring=[b]",
  literate =    {---}{{\ProcessThreeDashes}}3
                {>}{{\textcolor{red}\textgreater}}1
                {|}{{\textcolor{red}\textbar}}1
                {\ -\ }{{\mdseries\ -\ }}3,
}

% switch to key style at EOL
\lst@AddToHook{EveryLine}{\ifx\lst@language\language@yaml\YAMLkeystyle\fi}
\makeatother

\newcommand\ProcessThreeDashes{\llap{\color{cyan}\mdseries-{-}-}}

\lstdefinestyle{YAML}
{
    style=Common,
    language={yaml},
    morekeywords=
    {
        % add your new fortran keywords here!
    },
}

\lstnewenvironment{YAML}{\lstset{style=YAML}}{}

% !TEX root = ../Masterdatei.tex

\lstdefinestyle{CSharp}
{
    style=Common,
    language=[sharp]{c},
    alsolanguage={[LaTeX]TeX},
                keywordstyle=\color{blue},
                stringstyle=\color{red},
                commentstyle=\color{green},
                morecomment=[l][\color{magenta}]{\#},
    morekeywords=
    {
        % add your new fortran keywords here!
    },
}

% !TEX root = ../Masterdatei.tex

\lstdefinelanguage{JavaScript}{
  keywords={typeof, new, true, const, false, try, catch, instanceof, throw, function, return, null, switch, var, if, in, let, for,  while, do, else, case, async, await, break},
  keywordstyle=\color{blue}\bfseries,
  ndkeywords={class, export, boolean, implements, import, this},
  ndkeywordstyle=\color{magenta}\bfseries,
  identifierstyle=\color{black},
  sensitive=false,
  comment=[l]{//},
  morecomment=[s]{/*}{*/},
  commentstyle=\color{gray}\ttfamily,
  stringstyle=\color{orange}\ttfamily,
  morestring=[b]',
  morestring=[b]"
}

\lstdefinestyle{JavaScript}
{
    style=Common,
    language={JavaScript},
    morekeywords=
    {
        % add your new fortran keywords here!
    },
}

\lstnewenvironment{JavaScript}{\lstset{style=JavaScript}}{}

